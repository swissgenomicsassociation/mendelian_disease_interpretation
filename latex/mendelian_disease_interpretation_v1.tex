%%%% Document type  %%%%
%\documentclass[preprint,12pt,fleqn]{article}
\documentclass[12pt]{article}
\usepackage{ragged2e}

% \usepackage{nopageno} % no page numbers
\usepackage{placeins} % \FloatBarrier
\usepackage{upgreek, bm}

\usepackage[most]{tcolorbox}
\newtcolorbox[auto counter,number within=chapter]{definition}[1][]{
  enhanced,
  breakable,
  fonttitle=\scshape,
  title={Definition \thetcbcounter},
  #1
}

%%%% Document structure %%%%
%\usepackage{geometry}
\usepackage[verbose=true,letterpaper]{geometry}
\geometry{
    textheight=9in,
    textwidth=6in,
    top=1in,
    headheight=12pt,
    headsep=25pt,
    footskip=30pt,
}

\usepackage{lineno} % used along with \linenumbers after begin document. 
\usepackage{setspace} 
\setstretch{1.25}
\makeatletter % The following lines get rid of footer stating pre-preint to elsevier.
\def\ps@pprintTitle{%
\let\@oddhead\@empty
\let\@evenhead\@empty
\def\@oddfoot{}%
\let\@evenfoot\@oddfoot}
\makeatother
\graphicspath{ {../images/} }
\usepackage{pgf} % calculate cohort stats percentage

%%%% Bibliography   %%%%
\usepackage{natbib}
\setcitestyle{numbers,sort&compress}
\setcitestyle{sort&compress}
\usepackage{hypernat} 
    
%%%% Aesthetics     %%%%
\usepackage{microtype}
% \RequirePackage{times} % Font
\usepackage{ccaption}
\usepackage{siunitx}
\usepackage[T1]{fontenc}
\usepackage[utf8]{inputenc}
\usepackage{nameref}% this allows a reference be named, to print unnumbered references by their section name (used here for linking to Supplemental text in this case).

%%%% Paragraph Formatting %%%
% \setlength{\parindent}{0pt}
\setlength{\parindent}{15pt}
\setlength{\parskip}{6pt plus 2pt minus 1pt}

%%%% Supplemental labels%%%%
%Define command to start a supplemental section
%set the supplemental letter used for figures (e.g. Figure E1)
\newcommand{\beginsupplement}{%
        \setcounter{table}{0}
        \renewcommand{\thetable}{S\arabic{table}}%
        \setcounter{figure}{0}
        \renewcommand{\thefigure}{S\arabic{figure}}%
         }

%%%% Building tables%%%%
\usepackage{booktabs} % required for tables
\usepackage{rotating,tabularx} 
\newcolumntype{Z}{ >{\centering\arraybackslash}X } % defining table content layout per box
\usepackage{ltablex} % allow page break between lines in tabularx
% \usepackage{caption} \captionsetup{font=normalsize} % to set the caption size as normal even when table is tiny.
\usepackage{multirow}
\usepackage{pdflscape}

%%%% Colors %%%%
\usepackage{xcolor} 
\definecolor{natureblue}{RGB}{5,110,210}
    \usepackage[colorlinks]{hyperref} 
\AtBeginDocument{%this allows colours to chage from the defined elsearticle template.
\hypersetup{
    	colorlinks=true,
        linkcolor={natureblue},
    	citecolor={natureblue},
        filecolor=blue!50!black,
        urlcolor=cyan,
    	}}

\definecolor{kispiblack}{HTML}{333333}
\definecolor{kispidarkblue}{HTML}{023047}
\definecolor{kispidarkgreen}{HTML}{006666}
\definecolor{kispired}{HTML}{C70000}
\definecolor{kispilink}{HTML}{007DB8}%219EBC
% \color{kispi_black} %default
\definecolor{kispiblue}{HTML}{701A57}
% City sunset: https://www.color-hex.com/color-palette/40131
\definecolor{colorSUNSET1}{HTML}{eeaf61}
\definecolor{colorSUNSET2}{HTML}{fb9062}
\definecolor{colorSUNSET3}{HTML}{ee5d6c}
\definecolor{colorSUNSET4}{HTML}{ce4993}
\definecolor{colorSUNSET5}{HTML}{6a0d83}
\definecolor{natureblue}{RGB}{5,110,210}    
\usepackage{dirtree}  % Load the dirtree package

% command to use these colors and formatting; xspace for correct spacing including with punctuation marks.
\usepackage{xspace}
\newcommand{\variablesdarkgreen}[1]{\textbf{\textcolor{kispidarkgreen}{#1}}\xspace}
 
\usepackage{tocloft}  % Customizing the Table of Contents
\setcounter{tocdepth}{2}

%%%% Include code %%%%
% \usepackage{verbatim}
\usepackage{listings}
\lstset{
    basicstyle=\ttfamily\small,
    breaklines=true,
    postbreak=\mbox{\textcolor{red}{$\hookrightarrow$}\space}, % 
    breakatwhitespace=false,
    % frame=single,
    showstringspaces=TRUE, % Don't show spaces in strings as special characters
    tabsize=2, 
    language=sh 
}

\usepackage{fontspec}

\usepackage[printonlyused,withpage,nohyperlinks]{acronym}
\usepackage{tikz}
\usetikzlibrary{calc}
\usepackage{amsmath, amssymb}

% author affil
\usepackage{authblk}
\renewcommand\Authfont{\normalsize}
\renewcommand\Affilfont{\small}
\setlength{\affilsep}{1em}

%\usepackage{fontspec}
%\usepackage{titlesec}
%\setmainfont{SF Pro Text}


\begin{document}

\newcounter{myboxcounter}
\newcommand{\boxlabel}[1]{%
  \refstepcounter{myboxcounter}%
  \label{#1}%
}

\title{Swiss Genomics Association consensus guideline for evidence-based genomic variant interpretation in Mendelian disease}

% in your preamble, before \author:
\newcommand{\QUANT}{1}
\newcommand{\IPSNEO}{2}
\newcommand{\SGA}{3}
%\newcommand{\GHI}{2}
%\newcommand{\KISPIIMM}{3}
% \newcommand{\METAB}{4}
%\newcommand{\LEEDS}{5}

\author[\QUANT]{Quant Group} 
%\author[\GHI]{Simon Boutry}% \textsuperscript{†}}
%\author[\GHI]{Ali Saadat}% \textsuperscript{†}} \textsuperscript{†}}
%\author[\KISPIIMM]{Maarja Soomann}% \textsuperscript{†}} \textsuperscript{†}}
%\author[\KISPIIMM]{Johannes Trück}
%\author[\GHI]{Jacques Fellay}
\author[\IPSNEO]{Dylan Lawless *}
\author[\SGA]{the joint 
SGA committee}
% \textsuperscript{†}
\affil[\QUANT]{The quantitative omic epidemiology group.} 
\affil[\SGA]{The Swiss Genomics Association  commitee.}
%\affil[\GHI]{Global Health Institute, School of Life Sciences, École Polytechnique Fédérale de Lausanne, Switzerland.}
%\affil[\KISPIIMM]{Division of Immunology and the Children’s Research Center, University Children’s Hospital Zurich, University of Zurich, Zurich, Switzerland.}
%\affil[\LEEDS]{Leeds Institute of Rheumatic and Musculoskeletal Medicine, University of Leeds, Leeds, UK.}
\affil[\IPSNEO]{Department of Intensive Care and Neonatology, University Children's Hospital Zurich, University of Zurich, Zurich, Switzerland.}

\maketitle
% \justify
\raggedright

\begin{tcolorbox}[
  colback=red!5!white,
  colframe=red!60!black,
  title=\textcolor{white}{\textbf{DRAFT – NOT FINAL}},
  fonttitle=\large\bfseries,
  coltitle=white,
  left=2mm, right=2mm, top=1mm, bottom=1mm,
  boxrule=0.8pt,
  sharp corners,
  enhanced,
  title filled=true,
  colbacktitle=red!80!black
]
This document is a \textbf{live public working draft} and does not represent a completed or officially endorsed release.  
All statements, analyses, and opinions are those of the authors and have not yet undergone joint committee review or external peer review.  
Content is provided for transparency and collaborative development and may be revised substantially prior to publication.
\end{tcolorbox}
%\vspace{1em}
\clearpage

%  \linenumbers
\begin{abstract}
\noindent
% \vspace{1em}
Genomic medicine programmes across the world are advancing rapidly, especially with AI, but variant interpretation remains implemented differently across national, institutional and commercial efforts, with no shared agreement on how supporting evidence should be structured, recorded or exchanged. As a result, equivalent data are repeatedly recomputed, manually curated or privately remodelled, while shared learning, verification and reuse remain limited. Key evidence including inheritance, provenance, population context, functional data and conflicting observations is inconsistently captured, reducing transparency and interoperability and slowing collective progress.

\vspace{1em}
We define a harmonised, tool-agnostic specification that establishes a shared data architecture and scientific approach for variant interpretation in Mendelian disease. The guideline sets minimum requirements for representing, linking and auditing evidence, including sequence and sample provenance, variant normalisation, segregation logic, phenotype alignment, evidence grading, conflict handling and versioned synthesis statements. It emphasises structured, reviewable reasoning rather than fixed classification labels and is designed to complement, not replace, existing standards, tools and workflows.

\vspace{1em}
By formalising a standardised approach to quantifying interpretation, this framework enables distributed groups to solve genomics problems in alignment rather than in parallel isolation. It reduces duplication, supports independent validation, improves automation and ensures that advances in one sector directly strengthen others. The model establishes a shared, exchangeable foundation for interpretation that remains transparent, interoperable and evolvable, strengthening cooperation between national initiatives, healthcare, research and industry.
\footnote{
\noindent * Addresses for correspondence: \href{mailto:Dylan.Lawless@uzh.ch}{Dylan.Lawless@uzh.ch}.\\
% \noindent  \textsuperscript{† }These authors contributed equally and are listed alphabetically.\\
\textbf{Availability:} This data is integrated in 
\url{https://iei-genetics.github.io}.
%available under the MIT licence.
% \url{https://github.com/TheQuantGroup/QauntGroup}.
}
\vfill
\end{abstract}


%TEAM
%TEAM
%Ioannis
%Xenarios
%Chief Data Analytics Officer
%
%CONTACT
%Lorenzo Cerutti
%Bioinformatician
%
%CONTACT
%Arnaud Hungler
%Head of IT
%
%CONTACT
%Katrin Männik
%Head of Genomics Strategy
%
%CONTACT
%Ilya Kolpakov
%Data and Analytics Engineer
%
%CONTACT
%Arkadiy Shevrikuko
%Software Engineer
%
%CONTACT




\tableofcontents



% \section{Introduction}
% Set the motivation, problem definition, and need for a national evidence architecture.
%Genome sequencing is now routine in research, diagnostics, and commercial genomics, but the evidence supporting variant interpretations is recorded, structured, and justified in widely different ways. Algorithmic scores, proprietary rankings, or pathogenicity labels are often treated as conclusions rather than evidence, while key determinants of validity including sequence provenance, variant identity, inheritance logic, mechanism fit, population context, phenotype alignment, and contradictory observations are inconsistently captured or reported. This fragmentation limits reproducibility, weakens cumulative knowledge, and slows safe integration of AI-driven and cross-sector genomic innovation.
%
%The Swiss Genomics Association is building a national, open infrastructure for shared standards at the intersection of healthcare, research and industry. Our aim is not to replace existing frameworks or tools, but to define the evidential foundation they depend on: transparent, computable, and auditable variant interpretation that preserves biological nuance, remains interoperable across platforms, and retains full reasoning rather than only final labels. By establishing common expectations for evidence traceability, uncertainty, mechanistic reasoning, inheritance verification and phenotype grounding, this guideline enables consistent interpretation that can be exchanged, validated, and built upon by laboratories, companies, hospitals and researchers alike.
%
%This effort strengthens national coordination where common standards are still underdefined, ensuring genomic data remains clinically meaningful, scientifically reusable, and technically compatible with future care models, analytical innovation and AI-supported interpretation.

\section{Introduction}

Genome sequencing is widely used in clinical and research settings, yet the evidence used to justify variant interpretations is recorded and applied inconsistently. Many systems treat algorithmic scores or proprietary rankings as conclusions rather than inputs, and key evidence such as provenance, variant identity, inheritance fit, population context, phenotype alignment, and contradictory signals is often incomplete or incomparable across institutions. This limits reproducibility, complicates verification, and restricts the safe use of automated or AI supported interpretation.

The Swiss Genomics Association is developing a national evidence architecture that supports consistent interpretation across healthcare, research, and industry. The objective is not to replace existing pipelines or tools, but to define the minimal evidential contract they depend on. This contract specifies how evidence is structured, how uncertainty is recorded, and how falsification tests are applied to every proposed variant. It provides a transparent, computable basis for reasoning that can be traced, audited, and exchanged across systems.

By establishing shared expectations for variant identity, provenance, and evidence evaluation, the guideline enables interoperability, supports automation, and ensures that genomic data remain clinically meaningful and reusable across future analytical frameworks.

\section{Purpose and scope of the guideline}

This guideline defines the national minimum requirements for evidence based interpretation of variants in Mendelian disease. It sets out the computable rules and evidence flags that form the uniform evidence state for each variant. These rules apply falsification tests to all upstream outputs and record the presence, absence, or contradiction of supporting evidence.

The guideline focuses on how evidence is represented, computed, and interpreted. It does not prescribe sequencing workflows, vendors, prioritisation tools, or clinical reporting formats. Any institution may use its own pipelines, provided it supplies the minimal identifiers and evidence required for interoperability and reproducibility across the national system.


\section{Architectural context and national motivation}

A national genomics system must balance durability with adaptability. Clinical records, identifiers, and audit trails require structures that remain stable over decades, while biological knowledge, phenotype models, and interpretive resources evolve continuously. The guideline is designed for this mixed environment. It defines the stable elements needed for long term interoperability, including variant identity, provenance fields, and the structure of evidence states, while allowing evidence rules, knowledge sources, and computational logic to be updated without invalidating past interpretations. This separation ensures that new annotations, ontologies, or analytical frameworks can be incorporated as they mature, and that historical decisions remain traceable to the evidence that supported them at the time. The approach reflects the broader national strategy to build genomic infrastructure that is open, verifiable, and compatible with both public and private systems.

\section{Design principles for evidence-based variant interpretation}

The guideline is grounded in principles that support clarity, reproducibility, and transparent reasoning. Each evaluative step is defined as computable logic rather than as narrative instruction, ensuring that results can be reproduced exactly across institutions. Uncertainty is recorded explicitly so that missing or partial evidence is visible rather than inferred. Provenance links each flag to the rule, resource, and input that produced it, allowing complete reconstruction of decisions. The framework prioritises consistency across time and across analytic environments, enabling automation while maintaining clinical traceability. By treating variant interpretation as a set of falsification tests applied to each candidate, the system provides a stable evidential foundation that can be used reliably in clinical care, research, and industry without prescribing any particular workflow or toolchain.

\section{Data provenance and sample quality requirements}

Reliable interpretation depends on clear documentation of how each variant was generated. The system therefore requires a minimal set of provenance fields that describe the sample, the sequencing method, the reference build, and the primary variant calling pipeline. These identifiers do not constrain laboratories to specific technologies but ensure that downstream evaluation can be traced to defined inputs. Provenance is recorded as structured fields rather than free text so that evidence rules can reference them directly and so that analyses remain comparable across time and institutions.  

Sample quality must also be sufficient to support interpretation. Basic indicators such as coverage, call completeness, and genotype quality are required for the core falsification tests to operate correctly. Where quality or provenance information is missing, the system records explicit flags rather than inferring validity. This approach ensures that downstream conclusions reflect the trustworthiness of the inputs and that gaps in evidence remain visible to clinical reviewers and automated systems alike.


\section{Evidence rule framework}
% Explain the structure of rules and how they generate flags.

The evidence model in this guideline is based on reverse reasoning. For every variant proposed by an upstream tool, the system evaluates each evidence domain for signals that contradict, weaken, or fail to support the hypothesis that the variant explains disease. Each evaluation is implemented as a computable rule that inspects defined inputs and emits one or more evidence flags. Together, these rules produce a uniform evidence state for every variant that can be interpreted, exchanged, and audited in a consistent way.

Rules do not attempt to reproduce upstream prioritisation logic or clinical judgement. They operate on normalised variant identifiers, provenance fields, and standard annotation outputs, and they test whether key expectations are met. All outcomes, including missing or indeterminate results, are recorded explicitly as machine readable flags rather than left implicit in narrative reports or tool specific scores.

\subsection{Structure of evidence rules}
% Describe YAML rule definitions, version control, and how rules stay reproducible over time.

Each evidence rule is defined as a small, independent logical unit. In implementation, rules are encoded in a structured format such as YAML or JSON. They specify the required inputs, the conditions under which the rule can be applied, and the flags that are set when those conditions hold.

Inputs include normalised variant identifiers, population frequencies, inheritance information, functional annotations, phenotype mappings, and any necessary provenance or quality indicators. The rule then evaluates a clearly defined predicate, for example whether the allele frequency is incompatible with a rare Mendelian disorder given a reference dataset, or whether the observed segregation pattern contradicts the expected mode of inheritance. The result is one or more discrete flags that describe the evidence state, including counter evidence, uncertainty, or missing data.

Rules are designed to be composable and pipeline independent. Any laboratory or company can compute the same evidence state provided it supplies the required inputs and executes the rules as specified. This separation of logic from implementation allows the national standard to evolve without dictating local software choices.


\subsection{Versioning and audit trace}

To ensure reproducibility, every rule is associated with a versioned schema and a stable identifier. Rules are grouped into rule sets, which are released as numbered versions with defined validity periods. Each rule set can be referenced by a unique name and checksum so that any evidence state can be linked back to the exact logic that produced it.

When a rule is updated, for example to incorporate a new reference dataset or corrected threshold, the change is recorded as a new version rather than a silent modification. Historical rule sets remain available so that past interpretations can be reconstructed and compared with current behaviour. Audit trails therefore include both the flags assigned to each variant and the rule set version that generated them.

This versioning model applies equally to national recommendations and to local extensions. Institutions may define additional rules for internal use, provided these are clearly namespaced and versioned. The national minimum contract is defined by the core rule set; everything else is additive and does not alter the interpretation of those core flags.

 %\subsection{Optional reference implementation (QV framework)}
        % Old: QV file guideline content
        % Old: Implemented rule set (moved here conceptually)
        % Old: Rules under development (moved here conceptually) 
%Variant interpretation depends on how whole genome data are prepared and analysed. Each step, from sequencing and variant calling to annotation and filtering, defines what can be detected or missed. A qualifying variant (QV) protocol makes these steps explicit by describing the rules that determine variant inclusion and interpretation \cite{2025lawlessApplicationQualifyingVariants}.
%
%A standard QV protocol should specify the sequencing method, genome build, and tools used; the quality thresholds for coverage and genotype confidence; the genomic regions or panels considered; and the annotation or classification systems applied. Defining these parameters in a structured QV file separates logic from execution, allowing reproducible, auditable workflows. Each file has a version and checksum, linking it directly to analysis outputs.  
%
%This standardisation ensures that variant findings are traceable and comparable across studies. It also supports automated pipelines that integrate multiple evidence sources, such as population, functional, and multiomic data, in line with frameworks like the GA4GH Variant Annotation model. The result is consistent, transparent, and quantifiable variant interpretation.












\subsection{Optional reference implementation (QV framework)}
% Old: QV file guideline content, Implemented rule set, Rules under development

The guideline is accompanied by an optional reference implementation based on Qualifying Variant (QV) protocols \cite{2025lawlessApplicationQualifyingVariants}. A QV protocol encodes the complete evidence rule set in a single, machine readable file. It specifies the sequencing method, reference build, analysis tools, coverage and quality thresholds, genomic regions or panels, and the evidence rules applied to each variant. This configuration separates interpretive logic from software execution and allows identical workflows to be reproduced across sites.

QV files are expressed in YAML, with a simple key value structure that can be parsed by any modern programming environment. Each file includes metadata, version identifiers, and a checksum so that downstream outputs can be linked unambiguously to the protocol used. The same file can be used to drive both clinical and research analyses, ensuring that evidence rules are applied consistently and that any changes are transparent.

An excerpt from a QV interpretation prompt is shown in Listing~\ref{lst:qv_variant_interpretation}. In practice, full QV sets are constructed using a public builder tool and released with the corresponding rule set versions. Laboratories and companies are free to implement the national rules directly or to adopt the QV framework as a ready to use reference.

\lstinputlisting[
  style=qvstyle,
 caption={Excerpt from the QV interpretation prompt used with the QV Builder app. Line wrapping is shown for display only. For actual use, refer to the original source file or the corresponding official QV set release.},
  label={lst:qv_variant_interpretation}
]{../qv_variant_interpretaion.txt}


% \subsection{Implemented rule set}
% \subsection{QV set preparation} % Present; describes rule structure and builder tool

% Qualifying Variant (QV) sets define structured, reproducible variant interpretation rules using a transparent, machine-readable YAML format. Each QV file specifies metadata, filters, and evidence-based criteria in a simple key–value syntax, enabling precise replication of variant selection and interpretation logic across studies. QV sets can be composed interactively using the online \url{https://switzerlandomics.ch/pages/qv_builder/}, which provides an intuitive interface for defining rules, previewing YAML output, and ensuring consistency with established standards. This framework promotes FAIR data principles and harmonised variant interpretation for both research and clinical applications.

\subsection{Implemented rule set}

The implemented rule set defines the current national minimum contract for computable evidence. Each rule is encoded in a stable YAML structure and operates independently of any specific sequencing or annotation pipeline. The rules cover population frequency, inheritance and segregation, functional annotation, genotype validity, and quality related provenance checks. Together they produce the complete evidence state represented by the flag set defined in this guideline.

Qualifying Variant (QV) sets provide a practical reference implementation of these rules. A QV file specifies the rule versions, required inputs, and the conditions under which each flag is generated. The format is simple and machine readable, enabling consistent execution across institutions and software environments. QV sets can be assembled using the public builder at \url{https://switzerlandomics.ch/pages/qv_builder/}, which validates rule definitions, ensures correct structure, and allows users to preview the resulting YAML. This promotes transparent, reproducible configuration and supports FAIR principles by separating interpretive logic from local pipelines.

The implemented rule set is updated through versioned public releases. Older versions remain available for audit, comparison, and reconstruction of historical interpretation states. This ensures that interpretations can always be traced to the exact rules applied at the time and that changes to the national standard remain transparent and reviewable.


\section{Evidence flags}
% Introduce the complete flag set and their role in the framework.

Evidence flags represent the computable outcome of all evidence rules defined in this guideline. Each flag records whether a specific domain provides support, uncertainty, or counter evidence for a variant’s role in disease. The flags form a uniform, machine readable evidence state that can be compared and audited across institutions and over time.

\subsection{Evidence domains}

Each evidence domain tests whether the observable data are compatible with the hypothesis that a variant explains disease. These domains include population frequency, inheritance and segregation, molecular function, protein level consequence, genotype validity, and other external evidence sources. The system applies the same principle to all domains: if the available data provide a reason to question the hypothesis, a flag is set and recorded explicitly.

The appendix provides detailed descriptions of each domain, including the logic applied, examples of counter evidence, and extended illustrations for reference. The main text presents the domains only at conceptual level to maintain clarity.

\subsection{Final flag set}
% Present only the authoritative list.
% No examples, no tables, no figures here.
% Present the authoritative list of flags exactly as defined in the rule framework.
The following list defines the current authoritative set of national evidence flags. These flags are produced directly by the rule framework and represent the complete evidence state for each variant. Future versions of the rule set will be released publicly and are intended to remain back compatible so that older interpretations can be reproduced without modification. All technical details, examples, and rule definitions are provided in the supplemental appendix.

\begin{tcolorbox}[
    colback=white!0,
    colframe=black,
    boxrule=1pt,
    arc=1mm,
    outer arc=1mm,
    title=\textbf{\refstepcounter{myboxcounter}\label{box:ga4gh_va_example}Box \themyboxcounter: Flags}
]
\begin{verbatim}
flag_gt_valid
flag_moi_parent_gt_missing_mother
flag_moi_parent_gt_missing_father
flag_moi_parent_gt_missing_any
flag_moi_parent_gt_hom_mother
flag_moi_parent_gt_hom_father
flag_moi_parent_gt_hom_any
flag_moi_parent_conflict_AD
flag_moi_parent_conflict_AR
flag_moi_parent_conflict_XR
flag_moi_parent_conflict_any
flag_popfreq_common
flag_popfreq_rare
flag_popfreq_ultrarare
flag_missing_popfreq
flag_uniprot_hits_any_feature
flag_uniprot_hits_domain_like
flag_uniprot_hits_structural_like
flag_uniprot_hits_ptm_like
flag_uniprot_hits_binding_like
flag_uniprot_hits_variant_like
flag_uniprot_is_lof
flag_uniprot_predicted_nmd
flag_uniprot_truncates_feature
\end{verbatim}
\end{tcolorbox}



\subsection{Interpretation of flag combinations}

Individual flags describe the behaviour of specific evidence domains, but interpretation depends on their combined pattern. A variant may carry flags indicating missing evidence, technical uncertainty, biological inconsistency, or direct contradiction of an expected disease mechanism. The aggregation of these signals defines the overall evidential state used in Pillar 3 probabilistic reasoning.

Combinations are not reduced to a single classification. Instead, they form a transparent summary of all falsification attempts across domains. This preserves nuance, allows systematic reanalysis as new knowledge or rule versions emerge, and ensures that clinical interpretation remains grounded in explicit evidence rather than implicit assumptions.


\section{Structured representation and interoperability}
% Describe how evidence is encoded for exchange across systems.

Interoperability is essential for a national evidence framework that must operate across laboratories, hospitals, research groups, and commercial systems. The guideline therefore relies on established, internationally recognised data standards rather than bespoke formats. These standards ensure that variant identity, phenotypic descriptions, and evidence statements can be exchanged without loss of meaning and that automated systems can interpret and recompute evidence consistently across environments.


%\subsection{Alignment with GA4GH Variant Representation and Variant Annotation models}
%% Clarify how the framework aligns with GA4GH VRS and VA without duplicating their remit.
%
%% Justifies use of global standards for files (VCF, BAM/CRAM), phenotyping (HPO/ORDO), exchange (Phenopackets, Beacon), and reproducible compute (GA4GH-aligned tools and platforms)
%   
%%
%%
%%
%%only a handful of papers are essential to reference explicitly if the goal is to anchor your section in established, standards-based genomics rather than the broader prenatal literature.
%%
%%Here are the **core stand-out references** worth including, grouped by purpose:
%%
%%**1. Foundational standards and interoperability**
%%
%%* Jacobsen JOB et al., *Nat Biotechnol*, 2022 — Phenopacket schema (GA4GH).
%%* Rambla J et al., *Hum Mutat*, 2022 — Beacon v2 and federated data discovery.
%%* Robinson PN et al., *Am J Hum Genet*, 2008 — Human Phenotype Ontology (HPO).
%%* Pavan S et al., *PLoS One*, 2017 — Orphanet Rare Disease Ontology (ORDO).
%%* McKenna A et al., *Genome Res*, 2010 — GATK framework for standardised variant calling.
%%
%%**2. Phenotype-driven variant interpretation and standardised automation**
%%
%%* Smedley D et al., *Nat Protoc*, 2015 — Exomiser: integrated, phenotype-driven prioritisation.
%%* Robinson PN et al., *Am J Hum Genet*, 2020 — LIRICAL: interpretable likelihood-ratio framework.
%%* Kelly C et al., *Trends Genet*, 2022 — Overview of phenotype-aware prioritisation methods.
%%
%%**3. Aggregated data and clinical implementation exemplars**
%%
%%* 100,000 Genomes Project Pilot Investigators, *N Engl J Med*, 2021 — clinical implementation of large-scale rare disease genomics.
%%* Wright CF et al., *N Engl J Med*, 2023 — current benchmark for genomic diagnosis in rare paediatric disease.
%%
%%**4. Prenatal aggregation and ontology extension (for context, not methods)**
%%
%%* Duyzend MH et al., *Prenat Diagn*, 2024 — synthesis on standards and aggregation in prenatal genomics.
%%
%%A brief reference list citing only these papers will cover all necessary foundations: standard data models (GA4GH, HPO, ORDO), interoperable exchange (Beacon, Phenopackets), validated variant analysis frameworks (Exomiser, LIRICAL, GATK), and real-world evidence of aggregated, standardised implementation (100kGP, Wright et al., 2023).
%%
%%Anything beyond these would largely repeat or contextualise rather than expand the standards argument.
%
%% See \url{https://www.ncpi-acc.org} for NIH Cloud Platform Interoperability.
%
%% This section discusses VCF, BAM, HPO, ORDO, Phenopackets, Beacon, cloud pipelines, and Exomiser. It belongs in the interoperability section because it defines external standards the architecture depends on.
%Automation, interoperability, and reproducibility in genomics rely on the use of internationally recognised standards rather than isolated or proprietary tools. Validated frameworks enable traceable, comparable, and scalable handling of genomic and phenotypic data across research and clinical contexts \cite{2024duyzendImprovingPrenatalDiagnosis}.
%
%Core standards include the \textbf{Variant Call Format (VCF)}, \textbf{Binary Alignment Map (BAM/CRAM)}, and the \textbf{Phenopacket Schema} from the \textbf{Global Alliance for Genomics and Health (GA4GH)}. The \textbf{Human Phenotype Ontology (HPO)} and \textbf{Orphanet Rare Disease Ontology (ORDO)} support structured phenotyping and disease mapping, while the \textbf{Beacon protocol} enables federated search across distributed datasets without sharing raw data.
%
%Using such standards ensures long-term compatibility, supports automation, and improves accuracy through consistent validation and version control. Tools such as \textit{Exomiser}, \textit{LIRICAL}, and \textit{PhenIX} show how ontology-based approaches enable automated, phenotype-driven variant prioritisation. For cloud analysis, platforms like \textbf{Terra} and pipelines such as \textbf{GATK} provide transparent, reproducible environments aligned with GA4GH and international data-sharing frameworks.

\subsection{Alignment with GA4GH Variant Representation and Variant Annotation models}

This framework aligns with the \ac{ga4gh} standards for representing genomic variation and associated evidence. The \ac{vrs} defines computable, globally unique representations of sequence variation, enabling unambiguous exchange independent of file format or provider. The \ac{va} model specifies how evidence, conditions, and interpretive assertions can be encoded as structured objects with clear provenance. Together, these standards provide the semantic backbone for portable variant identity and portable evidence statements.

The guideline builds on this foundation by requiring that key identifiers and evidence outputs remain compatible with \ac{vrs} objects and \ac{va} evidence structures. This ensures that the national flag based system can interoperate with global infrastructures and that variant level assertions can be integrated into downstream applications, registries, or decision support systems.

This alignment is complemented by established community standards. \ac{vcf} and \ac{bam}/\ac{cram} remain the primary formats for variant calls and alignments, while structured phenotype and disease descriptors use the \ac{hpo} and \ac{ordo}. The \textit{Phenopacket} schema provides a GA4GH backed model for transmitting harmonised case level data, and the \textit{Beacon} protocol supports federated discovery across distributed datasets. These standards allow institutions to participate in national and international initiatives without modifying their core workflows. They also provide a consistent substrate for automated interpretation tools such as Exomiser, LIRICAL, and related phenotype driven frameworks.

%\subsection{GA4GH Variant Annotation model}
%% This matches the GA4GH VA explanatory content. It belongs in the interoperability layer, not in the evidence rule layer.  Introduces machine-interpretable evidence models (GA4GH VA), linking variant, condition, and evidence to support quantitative, updateable inference beyond manual ACMG categories
%
%See \url{https://va-spec.ga4gh.org/en/latest/examples/acmg-variant-pathogenicity-statement-with-evidence.html} for example. 
%
%Variant interpretation has traditionally relied on expert review and rule-based systems such as the ACMG/AMP criteria, where evidence is manually classified into discrete levels (e.g. “strong”, “moderate”, “supporting”). While this provides clinical transparency, it limits scalability and does not easily incorporate quantitative or experimental data generated outside the sequencing pipeline.  
%
%Structured annotation frameworks, such as the \textbf{GA4GH Variant Annotation (VA) specification}, provide a way to formalise how evidence supports or refutes pathogenicity, linking statements to their provenance and strength. Rather than a human-readable label alone, each assertion is represented as a computable object that connects a variant, a condition, and all supporting evidence lines. These may include cohort allele frequencies, functional assays, or other study results, each qualified by method, direction, and strength of support. This structure allows integration of diverse evidence types while retaining traceability to original data sources.  
%
%Such a model enables a shift from categorical to quantitative reasoning. A variant’s pathogenicity statement can accumulate weighted evidence from multiple domains: ACMG-derived criteria, \textit{in vitro} functional data (e.g. MAVE), RNA and protein evidence, population studies, or model organism data. Each evidence line can be standardised using controlled vocabularies and shared identifiers, allowing aggregation across studies and automated computation of posterior probabilities.  
%
%Tools such as \textit{Exomiser} and \textit{LIRICAL} illustrate early automation based on ACMG-compatible logic and phenotype-driven scoring. Extending these with the GA4GH VA model allows incorporation of continuous, probabilistic evidence rather than threshold-based categories. This approach transforms variant classification from interpretive judgment into quantifiable inference, reducing subjective variability and supporting automated re-evaluation as new evidence emerges.  
%
%By representing all evidence as structured, machine-interpretable data with defined provenance, the GA4GH framework provides a foundation for fully automatable, transparent, and continuously updatable variant interpretation.
%
%
%The Variation Representation Specification (VRS) is development by the GA4GH in the Genomic Knowledge Standards (GKS) Work stream;
%\url{https://www.ga4gh.org/product/variation-representation/}.

\subsection{GA4GH Variant Annotation model}

The \ac{va} model provides a structured approach for representing variant level assertions together with the evidence that supports or contradicts them. Traditional interpretation frameworks rely on human written summaries or categorical labels, which are difficult to recompute and cannot easily integrate quantitative or external data sources. In contrast, \ac{va} represents each assertion as a machine readable object linking the variant, the condition, the evidence lines, and their provenance.

This structure supports quantitative reasoning, where multiple evidence sources such as cohort frequencies, functional assays including MAVE, RNA or protein data, or model organism results can be encoded using controlled vocabularies and weighted consistently. It allows variant interpretations to evolve transparently as new evidence emerges, without losing the historical context required for audit or reproducibility.  

By ensuring compatibility with \ac{va}, the national framework enables each flag and evidence state to be incorporated into a wider ecosystem of automated reasoning tools and variant knowledge bases. The structured flag set can be referenced directly in \ac{va} evidence collections, and the falsification logic defined in this guideline can be embedded as computable rules beneath future probabilistic or decision support models.

The \ac{vrs}, developed within the \ac{gks} work stream, complements this by defining the core objects for representing alleles, haplotypes, and variations in a consistent and implementation independent manner. Together, \ac{vrs} and \ac{va} provide the global foundation for the structured, interoperable, and evolvable representation of variant evidence that this guideline depends on.

\subsection{Exchange formats and machine readable outputs}
% Define JSON, YAML, or other outputs that ensure interoperability.

The evidence framework is designed so that every output can be exchanged and reused without custom transformation. All rule definitions are distributed in \texttt{YAML} so that logic remains readable, versioned, and portable across programming environments. Variant level evidence states are produced as structured \texttt{JSON} objects that mirror the identifiers and semantics defined in \ac{vrs} and the \ac{va} specification. This ensures that flags, provenance fields, and supporting metadata can be consumed directly by clinical systems, research pipelines, registries, or future decision support tools.

The same representation applies across institutions. Each variant receives a stable identifier, a complete flag set, and a minimal provenance record encoded as machine readable fields rather than free text. Historical interpretations can be reconstructed by replaying the rule version associated with each output. This approach avoids fragmentation, supports long term compatibility, and allows independent implementations to reach identical conclusions from the same inputs.

\section{Probabilistic interpretation in Pillar 3}
  % Clarifies that current document is foundation for Pillar 3 logic
% Frame this work as the foundation for the probabilistic model used in Pillar 3.

\subsection{Relationship to Pillar 1 and Pillar 2 inputs}
% Clarify dependencies on provenance and normalised analysis variables.

\subsection{Evidence aggregation logic}
% Describe at a high level how evidence states will feed into probabilistic interpretation.

\section{Clinical review checkpoints}
% Define what automated rules cannot do and where expert review must occur.

\subsection{Required manual assessments}
% Provide a concise checklist of mandatory clinical oversight steps.

\subsection{Boundary of automated reasoning}
% State clear limits to prevent over-interpretation of algorithmic evidence.

\section{Reporting and synthesis statements}
% Define the structure and content of final variant reports based on flags and evidence.

\section{Auditability, versioning, and reproducibility}
% Describe logging, provenance, versioning, and how decisions can be reconstructed.

\section{Limitations and scope}
% Explicitly state what falls outside this guideline.

\section{Future extensions}
% Outline expected expansions without committing to specific timelines.




\section{Conclusion}


\section*{Acknowledgements}

\section*{Contributions}
DL designed the analyses and wrote the manuscript.

\section*{Competing interest}
\noindent
The authors declare no competing interest. 

\section*{Ethics statement}
\noindent
This study only used data which was previously published and publicly available, as cited in the manuscript.

\section*{Data availability}
The data used in this manuscript is derived from open sources which are cited in methods. The data generated is available from ...

\section*{Funding}

% \clearpage
\bibliographystyle{unsrtnat}
\bibliography{references}



\section*{Acronyms}
\renewenvironment{description}%
  {\list{}{\labelwidth0pt\itemindent-\leftmargin
    \parsep-1em\itemsep0pt\let\makelabel\descriptionlabel}}
  {\endlist}
\begin{acronym}
\acro{acmg}[ACMG]{American College of Medical Genetics and Genomics}%
\acro{acat}[ACAT]{Aggregated Cauchy Association Test}%
\acro{ad}[AD]{Autosomal Dominant}%
  \acro{af}[AF]{Allele Frequency}
 \acro{aid}[AID]{Autoinflammatory Disorders}
 \acro{anova}[ANOVA]{Analysis of Variance}
 \acro{ar}[AR]{Autosomal Recessive}
 \acro{bmf}[BMF]{Bone Marrow Failure}
 \acro{cd}[CD]{Complement Deficiencies}
 \acro{ci}[CI]{Confidence Interval}
 \acro{cri}[CrI]{Credible Interval}
 \acro{cid}[CID]{Immunodeficiencies affecting Cellular and Humoral Immunity}
 \acro{cid+}[CID+]{Combined Immunodeficiencies with Associated or Syndromic Features}
 \acro{cf}[CF]{Cystic Fibrosis}
 \acro{cftr}[\textit{CFTR}]{Cystic Fibrosis Transmembrane Conductance Regulator}
 \acro{cvid}[CVID]{Common Variable Immunodeficiency}
  \acro{dclre1c}[\textit{DCLRE1C}]{DNA Cross-Link Repair 1C}
 \acro{dbnsfp}[dbNSFP]{database for Non-Synonymous Functional Predictions}
 \acro{ge}[GE]{Genomics England} 
 \acro{gnomad}[gnomAD]{Genome Aggregation Database}
 \acro{grch38}[GRCh38]{Genome Reference Consortium Human Build 38}
\acro{gvcf}[gVCF]{genomic variant call format}
 \acro{hgvs}[HGVS]{Human Genome Variation Society}
\acro{hvnc}[HVNC]{HGVS Variant Nomenclature Committee}
\acro{hugo}[HUGO]{Human Genome Organisation}
 \acro{hpc}[HPC]{High-Performance Computing}
 \acro{hsd}[HSD]{Honestly Significant Difference}
 \acro{hwe}[HWE]{Hardy-Weinberg Equilibrium}
 \acro{iei}[IEI]{Inborn Errors of Immunity}
  \acro{ig}[Ig]{Immunoglobulin}
 \acro{il2rg}[\textit{IL2RG}]{Interleukin 2 Receptor Subunit Gamma}
 \acro{indel}[InDel]{Insertion/Deletion}
 \acro{iuis}[IUIS]{International Union of Immunological Societies}
 \acro{ld}[LD]{Linkage Disequilibrium}
 \acro{loeuf}[LOEUF]{Loss-Of-function Observed/Expected Upper bound Fraction}
 \acro{lof}[LOF]{Loss-of-Function}
 \acro{moi}[MOI]{Mode of Inheritance}
 \acro{nfkb1}[\textit{NFKB1}]{Nuclear Factor Kappa B Subunit 1}
 \acro{omim}[OMIM]{Online Mendelian Inheritance in Man}
 \acro{pad}[PAD]{Predominantly Antibody Deficiencies}
 \acro{pid}[PID]{Primary Immunodeficiency}
 \acro{pird}[PIRD]{Diseases of Immune Dysregulation}
 \acro{ppi}[PPI]{Protein-Protein Interaction}
 \acro{pli}[pLI]{Probability of being Loss-of-function Intolerant}
 \acro{qc}[QC]{Quality Control}
 \acro{rag1}[\textit{RAG1}]{Recombination activating gene 1}
 \acro{scid}[SCID]{Severe Combined Immunodeficiency}
 \acro{snv}[SNV]{Single Nucleotide Variant}
 \acro{skat}[SKAT]{Sequence Kernel Association Test}
 \acro{stringdb}[STRINGdb]{Search Tool for the Retrieval of Interacting Genes/Proteins}
 \acro{tp}[TP]{true positive}
\acro{fp}[FP]{false positive}
\acro{tn}[TN]{true negative}
\acro{fn}[FN]{false negative}
\acro{tnfaip3}[\textit{TNFAIP3}]{Tumor necrosis factor, alpha-induced protein 3}
 \acro{umap}[UMAP]{Uniform Manifold Approximation and Projection}
 \acro{uniprot}[UniProt]{Universal Protein Resource} 
 \acro{vcf}[VCF]{variant call format}
 \acro{vep}[VEP]{Variant Effect Predictor}
 \acro{vre}[VRE]{variant risk estimate}
  \acro{wgs}[WGS]{Whole Genome Sequencing}
 \acro{xl}[XL]{X-Linked}
\end{acronym}

%\\\\\\\\\\\\\\\\\\\\\\\\\\\\
\beginsupplement
\section{Supplemental} \label{Supplemental_text}

Supplemental data are presented under the same headings that correspond to their relevant main text sections. 


\section{Appendix}
% Holds supplementary materials relevant for implementation.

\subsection{Example rule files}
% Provide minimal YAML examples to clarify rule syntax.

\subsection{Example YAML or JSON evidence objects}
% Show how evidence flags appear in machine readable outputs.

\subsection{Worked examples of flagged variants}
% Provide illustrative examples for collaborators and reviewers.




\section{Evidence flags}
% Introduce the complete flag set and their purpose in the architecture.

\subsection{Evidence domains}
\subsubsection{Counter-evidence and conflict handling}
% Specify how benign signals, contradictory data, and mismatches are flagged and not hidden.

\subsubsection{Normalised variant representation and nomenclature}
    % National minimum input requirements
    % In QV these become flags such as flag_hgvs_valid
% Describe HGVS conventions, genome build, normalisation, and how variant identity is stabilised.
% Missing; should specify HGVS, VCF normalisation, reference builds, allele consistency, transcript rules

Accurate variant description depends on consistent use of internationally recognised nomenclature systems. The \ac{hgvs} nomenclature provides the authoritative standard for describing sequence variants at the DNA, RNA, and protein levels. It ensures that each variant is expressed unambiguously and reproducibly across clinical reports, publications, and databases.  

The \ac{hgvs} Nomenclature is maintained by the \ac{hgvs}   \ac{hvnc} under the H \ac{hugo}  and is widely implemented across major genomic resources and clinical interpretation platforms. Current recommendations are detailed in \citet{2024hartHGVSNomenclature2024} and \citet{2016dendunnenHGVSRecommendationsDescription}, which formalise the syntax, reference sequence alignment, and conventions for variant expression.  

Within this guideline, all variants should be reported according to the most recent \ac{hgvs}  Nomenclature release, aligned to an approved reference sequence (RefSeq or Ensembl transcript). Both coding (\texttt{c.}) and protein (\texttt{p.}) level annotations should be provided where possible. Genomic coordinates should follow the \ac{grch38} reference assembly.  

Using standardised nomenclature ensures interoperability between laboratories, software tools, and public databases, reducing ambiguity in variant exchange and supporting precise traceability in both research and clinical reporting.  

\subsubsection{ACMG criteria with counter-factual evidence} % Present but incomplete; should later define explicit additions for contradictory evidence handling

% By aggregating variant sets beyond single-gene analyses, the framework complements established rare variant association tests such as \ac{skat} and \ac{acat} \cite{liu2019acat,li2020dynamic,wu2011rare,lee2012optimal} and aligns with multi-omics integration methods \cite{kong2018nature,howe2021within}. It is compatible with \ac{acmg} guidelines \cite{richards2015standards}, related interpretation frameworks \cite{tavtigian2020fitting,li2017intervar}, and current \ac{qc} standards \cite{pedersen2021effective,anderson2010data}. Standardised systems, including \ac{acmg} Secondary Findings v3.2 \cite{miller2023acmg}, allow integration of these probabilistic results into clinical workflows. Comparison with AlphaMissense scores showed that pathogenicity predictors estimate molecular effect rather than probability of occurrence. Integrating variant-level priors with predictors like AlphaMissense could improve disease relevance by incorporating gene-disease associations  and \ac{moi} for applications in AI/ML.


The \texttt{acmg\_criteria} rule interprets the condensed ACMG criteria column for each variant and flags variants that carry benign evidence. When a variant has passed upstream prioritisation tools such as Exomiser, the ACMG criteria are typically recorded in a single column as a list of applied evidence codes (for example, \texttt{PVS1, PM2, PP3}, or \texttt{BA1}). The downstream rule inspects this condensed string to determine whether it contains any benign evidence codes, recognised by the presence of the letter ``B'' (for example, \texttt{BA1}, \texttt{BS1--BS4}, or \texttt{BP1--BP8}). Variants with such codes are retained but marked for review, similar to those carrying gnomAD quality flags. This ensures that variants with benign evidence are not automatically excluded but instead require explicit assessment or justification in the final clinical interpretation.

Rank based scoring such as that from Tavanian offers benefits but similarly can miss the presence of counter-factual evidence that a variant might be considered benign but remain prioritised by the presence of other pathogenic flags.

\begin{table}[H]
\small
\renewcommand{\arraystretch}{1.2}
\setlength{\tabcolsep}{4pt}
\centering
\begin{tabularx}{\linewidth}{l l l X}
\hline
\textbf{variant\_id} & \textbf{ACMG\_criteria} & \textbf{outcome} & \textbf{report\_action} \\
\hline
1 & PVS1, PM2, PP3 & pass & include normally \\
2 & PVS1, PM2, BA1 & pass (flagged) & include with note: ``contains benign ACMG evidence (BA1); review interpretation.'' \\
3 & BP4 & pass (flagged) & include with note: ``benign supporting evidence; verify consistency with phenotype.'' \\
4 & PS2, PM5 & pass & include normally \\
\hline
\end{tabularx}
\caption{Example application of the \texttt{acmg\_criteria} rule to patient variants. Variants containing benign ACMG evidence codes are retained but flagged for explicit review in the final report.}
\label{tab:acmg_criteria}
\end{table}



\subsubsection{Population frequency evidence}
% Define frequency thresholds, rarity logic, missingness flags, and how population data enter interpretation.
\subsubsection{Conditional evidence rules and source-dependent quality checks} % Present; covers conditional database flags (gnomAD example)
% This is fine, but the gnomAD conditional rules are partly part of the flag system, not strictly population frequencies.

Reference databases and the case sample may share variant-calling or sequencing biases. Therefore, databases which carry flagged variants should be examined through conditional automation or manual review. 
For instance, GnomAD is a key reference for interpreting variants in single-case analyses.
The \texttt{gnomad\_flags} rule ensures that flagged variants are reviewed rather than excluded. Variants without a flag (\texttt{NA}) pass directly, while those with recognised gnomAD flags pass with review status.

Common flags include \texttt{AS-VQSR} (allele-specific quality recalibration), \texttt{RF} (random forest outlier), \texttt{LC pLoF} (low-confidence loss-of-function), and \texttt{SEGDUP} (segmental duplication). These indicate potential technical or annotation uncertainty rather than confirmed artefacts.
We recommend that variants are retained in reporting but require justification or comment to ensure transparency and traceability.


\begin{table}[H]
\small
\renewcommand{\arraystretch}{1.2}
\setlength{\tabcolsep}{4pt}
\centering
\begin{tabularx}{\linewidth}{l l l X}
\hline
\textbf{variant\_id} & \textbf{gnomad\_flag} & \textbf{outcome} & \textbf{report\_action} \\
\hline
1 & NA & pass & include normally \\
2 & LC pLoF & pass (flagged) & include with note: ``gnomAD LC pLoF; review interpretation.'' \\
3 & UnknownFlag & fail & hold for manual review \\
\hline
\end{tabularx}
\caption{Example application of the \texttt{gnomad\_flags} rule to patient variants. Entries automatically wrap within the column width for compact layout.}
\label{tab:gnomad_flags}
\end{table}

\subsubsection{Inheritance and segregation evidence}
% Present trio logic, zygosity checks, parental genotype completeness, and mechanistic compatibility.

Version 1 

When pedigree data are available, \ac{wgs} enables direct evaluation of inheritance models for each variant. Genotypes are interpreted across proband and parents using standard representations such as \texttt{REF}, \texttt{HET}, and \texttt{HOM}, or equivalent encodings used in variant data formats (\texttt{0,1,2} in PLINK, \texttt{0/0}, \texttt{0/1}, \texttt{1/1}, or phased forms such as \texttt{0|1} in VCF). From these data, the inheritance pattern for each variant is determined, such as de novo, homozygous, heterozygous, or compound heterozygous, and this information is then evaluated in relation to the known gene–disease relationship.
For example, the observed segregation pattern may provide supporting or contradictory evidence for an\ac{ad}, \ac{ar}, or \ac{xl} disease mechanism.

Where genotype data are incomplete, inheritance may be inferred from clinical features, family history, or segregation information, but such cases are explicitly flagged as uncertain. Scenarios of incomplete penetrance, such as a heterozygous variant inherited from an unaffected parent, are also recorded because they influence the strength of causal interpretation.
 Each inheritance assessment includes both the inferred pattern and the type of supporting evidence, ensuring that interpretative conclusions in the genetic report transparently reflect the available data and its confidence level.

Version 2

Accurate interpretation of inheritance requires integrating two complementary sources of evidence:  
(1) the \textit{\ac{moi}} defined by curated reference datasets that describe known gene–disease mechanisms, and  
(2) the \textit{observed inheritance pattern} derived from family genotype or clinical data.  

The reference \ac{moi} defines the expected transmission mechanism for a gene–disease pair, typically \ac{ad}, \ac{ar}, or \ac{xl}. Structured datasets such as PanelAppRex \cite{lawless_panelapprex_2025} harmonise these annotations across thousands of curated panels. Foundational sources including Genomics England’s PanelApp and PanelApp Australia \cite{martin_panelapp_2019} provide continuously updated expert curation underpinning national infrastructures such as the NHS National Genomic Test Directory and the 100,000 Genomes Project. The \ac{moi} field thus serves as an evidence-based prior — a quantitative expectation of how pathogenic variants in a given gene are likely to segregate.

The observed inheritance pattern, by contrast, is determined from the case data. When trio or family \ac{wgs} is available, inheritance can be assessed directly from genotype encodings (\texttt{REF}, \texttt{HET}, \texttt{HOM}, or \texttt{0/1}, \texttt{1/1}, \texttt{0|1}). This pattern may confirm or contradict the reference \ac{moi}. For example, a de novo heterozygous variant in an \ac{ad} gene supports causality, whereas biallelic variants in an \ac{ar} gene are expected.  

Cases with incomplete genotype data require inference from clinical or segregation information and must be explicitly flagged as uncertain. Incomplete penetrance, mosaicism, or unaffected carriers (e.g. heterozygous variants in \ac{ar} genes) should be documented, as they influence the posterior probability of pathogenicity \cite{lawless_quantifying_2025}.  

To ensure consistency, each variant interpretation should record both the reference \ac{moi} (from curated databases) and the observed inheritance pattern (from patient data). This dual recording enables probabilistic interpretation frameworks, such as Quant \cite{lawless_quantifying_2025}, to integrate population frequencies, genotype configurations, and inheritance priors under Hardy–Weinberg equilibrium. Together, these components quantify diagnostic confidence and prevent misclassification of variants arising from uncertain or incomplete pedigree information.



\subsubsection{Functional and molecular evidence}
% Describe UniProt feature mapping, domain overlap, truncation patterns, and molecular consequence rules.

Functional evidence from UniProt is integrated by detecting positional overlap between variant amino acid coordinates and annotated protein features recorded in UniProt GFF files. Each UniProt entry provides structured annotations describing biochemical, structural, and functional properties of the protein. For each variant, the affected residue position is compared against these annotated intervals, and any intersection is recorded as supporting evidence in the QV interpretation framework. This approach ensures that experimental and curated protein-level information contributes directly to variant interpretation and reporting (\textbf{Figure ~\ref{fig:tnfaip3_evidence_tracks}}).

The annotated features used as evidence sources include catalytic and binding sites, metal and nucleotide binding regions, and other experimentally defined functional motifs. Structural features such as helices, beta strands, and coiled coils provide spatial context for potential conformational disruption. Domain- and family-level annotations, including domains, motifs, and topological regions, capture conserved structural organisation and functional domains. Additional layers include post-translational modification sites, mutagenesis data, and known sequence variants curated in UniProt. Processing and localisation signals (such as signal peptides, transit peptides, and cleavage products) and cautionary sequence annotations (for example, frameshifts or sequence uncertainty) are also recorded.

By systematically linking these feature classes to variant coordinates, the framework records not only where functional or structural evidence exists, but also the type of information present—whether experimental, inferred, or computational. This enables each variant interpretation to transparently reflect the available molecular evidence supporting its classification.

\begin{figure}[ht]
\centering
\includegraphics[width=0.85\textwidth]{functional_evidence_uniprot.pdf}
\caption{
\textbf{Functional evidence tracks from UniProt annotations.}
The example illustrates how protein-level features such as domains, motifs, and catalytic sites provide structured evidence supporting interpretation of coding variants. 
Overlaps between variant positions and curated functional regions indicate potential mechanistic relevance, while the absence of overlap suggests limited or indirect evidence. 
This evidence framework guides the strength of interpretation in clinical reporting, ensuring that well-supported variants are highlighted and uncertain findings are transparently qualified.
}
\label{fig:tnfaip3_evidence_tracks}
\end{figure}



\subsubsection{Phenotype and gene–disease validity evidence}
% Present phenotype fit, MOI priors, gene validity rules, and disease spectrum alignment.


\end{document}
